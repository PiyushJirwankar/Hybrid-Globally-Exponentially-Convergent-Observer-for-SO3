\documentclass{article}
\usepackage[utf8]{inputenc}
\usepackage{amsmath, amsfonts, amssymb, mathtools, dsfont, xcolor, amsthm}
\usepackage{graphics}
\DeclareGraphicsExtensions{.pdf,.png,.gif,.jpg}
\usepackage{caption}
\usepackage{subcaption}
\newcommand{\SOtwo}{\mathbb{SO}(2)}
\newcommand{\sotwo}{\mathfrak{so}(2)}
\newcommand{\R}[1]{\mathbb{R}^{#1}}
\newcommand{\dotR}{\dot{R}}
\newcommand{\Omegay}{\Omega^y}
\newcommand{\vex}[1]{\text{vex}\left(#1\right)}
\newtheorem{theorem}{Theorem}
\newtheorem{assumption}{Assumption}
\newtheorem{definition}{Definition}
\newtheorem{lemma}{Lemma}
\newcommand{\trace}[1]{\text{tr}\left(#1\right)}
\newcommand{\brackets}[1]{\left(#1\right)}
\newcommand{\adjoint}[2]{\text{Ad}_{#1}#2}
\newcommand{\textblue}[1]{\textcolor{blue}{#1}}
\newcommand{\Rtilde}{\tilde{R}}
\newcommand{\etabound}{\Bar{\eta}_w}


\title{\textcolor{blue}{Hybrid Globally Asymptotically Stable Observer for $\SOtwo$}}
\author{Piyush Jirwankar}
\date{December 2022}

\begin{document}

\maketitle

The purpose of this work is to design a globally asymptotically stable observer for a system evolving on $\SOtwo$. \textblue{This article works on a nominal case where measurements are not corrupted with any noise.}

\section{Preliminaries}

Let the state of this system be $\textcolor{blue}{R\in\SOtwo}$, which can be parameterized with $\theta\in\mathbb{S}^1$ as \[\textcolor{blue}{R = \begin{bmatrix}
    \cos(\theta) && -\sin(\theta)\\
    \sin(\theta) && \cos(\theta)
\end{bmatrix}}\]
Let the Lie algebra of $\SOtwo$ be represented by $\sotwo$. We define the cross map as $(\cdot)_\times : \R{} \to \sotwo$, \[\textcolor{blue}{ \R{} \ni\omega\mapsto \omega_\times \coloneqq \begin{bmatrix} 0 && -\omega \\ \omega && 0\end{bmatrix}, \quad \forall \omega\in\R{}}\]
The inverse operator is defined as $\text{vex}: \sotwo \to \R{}$,\[\textcolor{blue}{\R{}\ni\theta_\times\mapsto \vex{\theta_\times}\coloneqq \theta, \quad \forall \theta_\times \in \sotwo}\]
Next, the exponential map is defined as $\exp: \sotwo \to \SOtwo$, \[ \theta_\times \mapsto \exp\left(\theta_\times \right)\coloneqq \begin{bmatrix}
    \cos(\theta) && -\sin(\theta)\\
    \sin(\theta) && \cos(\theta)
\end{bmatrix}\]
\textblue{We also define a map $\mathbb{P}_a : \SOtwo \to \sotwo$ such that \[\SOtwo\ni R \mapsto \mathbb{P}_a(R) \coloneqq \frac{1}{2}\brackets{R - R^T}\]}
The rotation kinematics on $\SOtwo$ are given below, where $\Omega$ represents the angular velocity in the body fixed frame. 
\begin{align}\label{eq:rotationKinematics}
    \dotR = R\Omega_\times
\end{align}
% One of the standard tools for estimation on the special orthogonal groups is the nonlinear complementary filter \cite{mahony_complementaryFilter}. The measurements of the angular velocity $\Omega$ and the rotation matrix $R$ are given as $\Omegay$ and $R^y$ respectively. The following assumption is made. 
% \begin{assumption} \label{ass:no_meas_noise}
%     The measurements are assumed noise-free and angular velocity measurements are assumed bias-free i.e. \textblue{$\Omegay=\Omega$ and $R^y = R$.}
% \end{assumption}
% \textblue{This article considers the nominal case where we have perfect measurements of $R$ and $\Omega$. The next step in this study would be to consider bounded measurement errors for $R^y$ and $\Omega^y$.}

% \textblue{Before moving on to the observer design, we define asymptotic independence of a pair of signals.}
% \textblue{
% \begin{definition}(Asymptotically Independent Signals \cite{mahony_complementaryFilter})
% Two signals $x(t) : \R{} \to M_x$ and $y(t) : \R{} \to M_y$ are \emph{asymptotically dependent} if there exists a nondegenerate (full ranked Hessian) function $f : M_x\times M_y \to \R{}$ and a time $T>0$ such that \[f(x(t), y(t)) = 0, \quad \forall t > T\]
% The two signals are asymptotically independent if they are not asymptotically dependent.      
% \end{definition}
% }
\textblue{Next, we define the local exponential stability and unstable sets for a dynamical system on $\SOtwo$.
\begin{definition}{(Local Exponential Stability)}
For a linear approximation of a rotation matrix $\SOtwo\ni R_1 \approx I + {x_1}_\times$ and $\SOtwo\ni R_2 \approx I + {x_2}_\times$, where $x_1, x_2\in \R{}$ with $e \coloneqq x_1 - x_2$, $R_1$ is locally exponentially stable to $R_2$ if there exists an $\alpha > 0$, and for every $\epsilon >0$, $\exists \delta (\epsilon) > 0$ such that \[|e(t)| = \epsilon \exp(-\alpha t), \quad \forall t > 0\] when $|e(0)| < \delta(\epsilon)$.
\end{definition}}
\textblue{\begin{definition}(Unstable set)
The unstable set $W_u$ of an invariant set $\mathcal{I}$ is defined as 
\[W_u(\mathcal{I}) \coloneqq\{ x\notin \mathcal{I} : x(t) \in \mathcal{I} \text{ as } t \to -\infty\}.\]
\end{definition}
}
\section{Observer design}

One of the standard tools for estimation on the special orthogonal groups is the nonlinear complementary filter \cite{mahony_complementaryFilter}. The measurements of the angular velocity $\Omega$ and the rotation matrix $R$ are given as $\Omegay$ and $R^y$ respectively. The following assumption is made. 
\begin{assumption} \label{ass:no_meas_noise}
    The measurements are assumed noise-free and angular velocity measurements are assumed bias-free i.e. \textblue{$\Omegay=\Omega$ and $R^y = R$.}
\end{assumption}
\textblue{This article considers the nominal case where we have perfect measurements of $R$ and $\Omega$. The next step in this study would be to consider bounded measurement errors for $R^y$ and $\Omega^y$.}

\textblue{Before moving on to the observer design, we define asymptotic independence of a pair of signals.}
\textblue{
\begin{definition}(Asymptotically Independent Signals \cite{mahony_complementaryFilter})
Two signals $x(t) : \R{} \to M_x$ and $y(t) : \R{} \to M_y$ are \emph{asymptotically dependent} if there exists a nondegenerate (full ranked Hessian) function $f : M_x\times M_y \to \R{}$ and a time $T>0$ such that \[f(x(t), y(t)) = 0, \quad \forall t > T\]
The two signals are asymptotically independent if they are not asymptotically dependent.      
\end{definition}
}

Now, we re-state below the required theorem from \cite{mahony_complementaryFilter} as applied to systems evolving on $\SOtwo$.


\begin{theorem} \label{thm:mahony}
(Theorem 4.2, \cite{mahony_complementaryFilter}) Consider the state $R\in\SOtwo$ that follows the kinematics in equation \eqref{eq:rotationKinematics}. The measurements satisfy Assumption \eqref{ass:no_meas_noise}. Let the estimate of $\textcolor{blue}{R}$ be denoted by $\textcolor{blue}{\hat{R}\in\SOtwo}$. \textcolor{blue}{Let $k_p > 0$ be a positive gain value. With the initial estimate being $\hat{R}(0) = \hat{R}_0 \in \SOtwo$}, let the estimates satisfy the following \textcolor{blue}{kinematics:}

\begin{subequations}\label{eq:observer}
\begin{align}
    \dot{\hat{R}} = \hat{R}\left( \Omegay + k_p\omega\right)_\times, \quad \hat{R}(0) = \hat{R}_0 \label{eq:}\\
    \omega = \vex{\mathbb{P}_a(\textcolor{blue}{\Tilde{R}^y})}, \quad \textcolor{blue}{\Tilde{R}^y \coloneqq \hat{R}^TR^y}
\end{align}
\end{subequations}
Define the estimation error as $\Tilde{R} = \hat{R}^TR$. \textcolor{red}{Assume that $\Omega(t)\in \R{}$
is a bounded}, absolutely continuous signal and that the pair of signals \textblue{$(\Omega(t), \Tilde{R}(t))$ are asymptotically independent}. Define $\mathbb{U}_0 \subseteq \SOtwo$ as 
\[\mathbb{U}_0 = \left\{ \Tilde{R}\in \SOtwo \mid \text{tr}(\Tilde{R}) = -2\right\}\]
Then:
\begin{enumerate}
    \item The set $\mathbb{U}_0$ is forward invariant and {unstable} with respect to the system given in equation \eqref{eq:observer}.
    \item {The error $\Tilde{R}$ is locally exponentially stable to the identity element $I$.} 
    \item For all initial conditions such that $\textblue{\hat{R}_0^TR^y(0) \notin \mathbb{U}_0}$, $\hat{R}(t)$ converges to $R(t)$.
\end{enumerate}
\end{theorem}
Using Theorem \ref{thm:mahony}, we have an almost globally asymptotically stable observer, with the unstable set being the measure zero set of $\SOtwo$ that represents a $180^\circ$ rotation. We wish to leverage this by using hybrid control theory to obtain a globally asymptotically stable observer on $\SOtwo$. 

\begin{definition}(Hybrid Observer)
    A hybrid observer $\hat{\mathcal{H}}$ with data $(C_K, F_K, D_K, G_K, \zeta)$.

\begin{align}
    \hat{\mathcal{H}} :  \begin{cases}
        \dot{\eta} = F_K(\eta, v), \quad (\eta, v) \in C_K\\
        \eta^+ = G_K(\eta, v), \quad (\eta, v)\in D_K\\
        \zeta = f(\eta, v)
    \end{cases}
\end{align}
where $\eta$ is the state of the hybrid observer, $v$ is the input and $\zeta$ is the output, $C_K$ and $D_K$ represent the flow set and the jump set for the hybrid observer respectively. 
\end{definition}


Define a potential function on $\SOtwo$ as below. 

\textblue{
\begin{definition}\label{def:config_err}
    The configuration error function for $\SOtwo$ is defined by $\Psi: \SOtwo \to \R{}$, \[\Psi({\Tilde{R}}) \coloneqq \frac{1}{2}\trace{I - \Tilde{R}}.\]
\end{definition}}
\textblue{
\begin{lemma} 
    The configuration error function $\Psi$ defined in Definition \ref{def:config_err} is positive definite, and the critical points of $\Psi$ are given by $\Theta =  \{I, \mathbb{U}_0\}$.
\end{lemma}
}


\textblue{Due to assumption 1, we have $R^y = R$}. Our design ideology for the globally asymptotically stable observer is the following: when \textblue{$\Tilde{R}^y = \tilde{R} \in \SOtwo \backslash \mathbb{U}_0$}, the observer is defined by equation \eqref{eq:observer}. For $\textblue{\Tilde{R}^y\in\mathbb{U}_0}$, an open loop observer is defined such that $\Tilde{R}$ escapes the unstable set $\mathbb{U}_0$. Thus, what we desire is the following.
\begin{align}
    \dot{\Tilde{R}} &= \Tilde{R} \underbrace{\begin{bmatrix}
        0 && 1\\
        -1 && 0
    \end{bmatrix}}_{\kappa_1} \nonumber\\
    &= \Tilde{R}\kappa_1 \label{eq:Rtilde_dyn}
\end{align}
Since we have assumed perfect measurement, we have the knowledge of the true rotation matrix $R$. Thus, using $\dot{\Tilde{R}} = \dot{\hat{R}}^TR + \hat{R}^T\dot{R}$ we can write \eqref{eq:Rtilde_dyn} as follows. 
\begin{align}
    \Tilde{R}\kappa_1 &= \dot{\hat{R}}^TR + \hat{R}^T\dot{R}\nonumber\\
    &= \dot{\hat{R}}^TR + \hat{R}^TR\Omega_\times\nonumber\\
    \implies \dot{\hat{R}}^TR &= \Tilde{R}\left(\kappa_1 - \Omega_\times\right)\nonumber\\
    \dot{\hat{R}}^T &= \Tilde{R}\brackets{\kappa_1 - \Omega_\times}R^T\nonumber\\
    \implies \dot{\hat{R}} &= R\brackets{\Omega_\times - \kappa_1}R^T\hat{R} \nonumber\\
    &= \hat{R}\Tilde{R}\brackets{\Omega_\times - \kappa_1}\Tilde{R}^T\nonumber\\
    &= \hat{R}\adjoint{\tilde{R}}{\left(\Omega_\times - \kappa_1\right)}\nonumber\\
    &= \textblue{\hat{R}\adjoint{\tilde{R}^y}{\left(\Omega_\times - \kappa_1\right)}\label{eq:observer_unstable}}
\end{align}
Note that the $\text{Ad}$ operator here represents the Adjoint map $\text{Ad}: \SOtwo\times \sotwo \to \sotwo$ \textblue{\[ \SOtwo\times \sotwo \ni(R_1, v_1) \mapsto \adjoint{R_1}{v_1} \coloneqq R_1 v_1 R_1^T \in \sotwo\]}

Thus, we have the observer for the unstable region in equation \eqref{eq:observer_unstable}. It appears that employing this observer in $\mathbb{U}_0$, and employing \eqref{eq:observer} in $\SOtwo\backslash \mathbb{U}_0$ should ensure globally asymptotic stability of the observer. However, such an observer could be very sensitive to small noise, that can result in chattering. Hence, we employ a hysteresis based observer. Consider two constants $ 0 < c_1 < c_0 < 2$. Consider the following sets that are functions of $\textblue{R^y}$.
\begin{align}
    C_0(\textblue{R^y}) &\coloneqq \left\{\hat{R} \in \SOtwo\mid \Psi\left(\hat{R}^T\textblue{R^y}\right) \leq c_{0} \right\}\\
    C_1(\textblue{R^y}) &\coloneqq \left\{\hat{R} \in \SOtwo\mid \Psi\brackets{\hat{R}^T\textblue{R^y}} \geq c_1 \right\}\\
    D_0(\textblue{R^y}) &\coloneqq \left\{\hat{R} \in \SOtwo\mid \Psi\left(\hat{R}^T\textblue{R^y}\right) \geq c_{0} \right\}\\
    D_1(\textblue{R^y}) &\coloneqq \left\{\hat{R} \in \SOtwo\mid \Psi\brackets{\hat{R}^T\textblue{R^y}} \leq c_1 \right\}
\end{align}
The intersection $\textblue{\Gamma({R}^y) \coloneqq C_0({R}^y)\cap C_{1}({R}^y)}$ represents the hysteresis region. The hybrid observer $\hat{\mathcal{H}}$ can be represented by the states $q \in Q\coloneqq \{0, 1\}$. The inputs for $\hat{\mathcal{H}}$ are $v = (v_1, v_2, v_3) \coloneqq (\hat{R}, \textblue{R^y}, \Omega^y) \in \SOtwo\times\SOtwo\times\R{}$, the output is $\zeta \in \sotwo$ which is defined below. Define $\gamma_0(v) \coloneqq (\Omega^y + k_p\omega)_\times$, $\gamma_1(v) \coloneqq \adjoint{\textblue{\Tilde{R}^y}}{\brackets{\Omega_\times - \kappa_1}}$. The data for the hybrid observer $\brackets{C_K, F_K, D_K, G_K, \zeta}$ is given as follows:
\begin{align}
    C_K &= \bigcup_{q\in Q}\brackets{C_{K, q}\times q}, \quad \begin{cases}
        C_{K, 0} \coloneqq C_0(\textblue{R^y})\\
        C_{K, 1} \coloneqq C_1(\textblue{R^y})
    \end{cases}\\
    F_K(v, q) &= 0\\
    D_K &= \bigcup_{q\in Q} \brackets{D_{K, q}\times Q}, \quad \begin{cases}
        D_{K,0} \coloneqq D_0(\textblue{R^y})\\
        D_{K,1} \coloneqq D_1(\textblue{R^y})
    \end{cases}\\
    G_K(v, q) &= 1 - q\\
    \zeta(v, q) &= q\gamma_1(v) + (1-q)\gamma_0(v)
\end{align}
Using the above hybrid observer, the estimates are updated as 
\begin{align} \label{eq:observer_final}
\underbrace{\dot{\hat{R}} = \hat{R}\zeta(v, q), \quad \dot{q} = 0}_{\text{during flows}}, \quad \text{ and}\quad  \underbrace{\hat{R}^+ = I, \quad q^+ = 1 - q}_\text{during jumps}
\end{align}

Consider the following hybrid system $\mathcal{H} = (C, F, G, D)$ with the states $(\Tilde{R}, q) \in \SOtwo\times Q$ has the following data. 
\begin{subequations}\label{eq:hybrid_observer_error}
\begin{align}
C&\coloneqq C_K\\
F(\Tilde{R}, q)&\coloneqq \begin{bmatrix}
    \Tilde{R}\brackets{\Tilde{R}^T\zeta(v, q)^T\Tilde{R} + \Omega_\times}\\ 0\end{bmatrix}\\
D_K &\coloneqq D_K\\
G_K(\Tilde{R}, q) &\coloneqq \begin{bmatrix}
\Tilde{R} \\ 1 - q
\end{bmatrix}
\end{align}
\end{subequations}

\begin{theorem}(Hybrid Observer Stability)
    Given the system \eqref{eq:rotationKinematics}, and consider the set $\mathcal{A}_1\coloneqq \{I\}\times \{0\} \in \SOtwo\times Q$, the following holds:
    \begingroup
    \renewcommand\labelenumi{(\theenumi)}
    \begin{enumerate}
        \item \label{(1)}The hybrid observer error system $\mathcal{H}$ with data \eqref{eq:hybrid_observer_error} satisfies the hybrid basic conditions.
        \item \label{(2)}Every maximal solution to $\mathcal{H}$ from $C\cup D$ is complete. 
        \item \label{(3)}The set $\mathcal{A}$ is globally asymptotically stable for $\mathcal{H}$. 
    \end{enumerate}
    \endgroup
    \end{theorem}

    \begin{proof}
         Since $C_0, C_1, D_0, D_1$ are closed subsets of $\SOtwo$ and the maps $F$ and $G$ are single valued and continuous on $C$ and $D$ respectively , \eqref{eq:hybrid_observer_error} satisfies the hybrid basic conditions. This proves $\eqref{(1)}$.
    
    Next, we note that $F(\tilde{R}, q)\in T_{\tilde{R}}\SOtwo$ for all $\tilde{R}\in \SOtwo$. Thus, using Proposition 6.10 from  \cite{hybridDynamicalSystems}, the proof of $\eqref{(2)}$ is complete. 
    
 We note that when $\tilde{R} = I$ and $q = 0$, $(\tilde{R}, q) \in \mathcal{A}_1$. If $q = 1$ and $\tilde{R} = I$, $(\tilde{R}, q) \in D_1$. Using the jump map, we have that $(\tilde{R}^+, q^+) \in\mathcal{A}_1$. Thus, it suffices to show that the set $\mathcal{A} \coloneqq
  \{I\} \in \SOtwo$ is globally asymptotically stable for $\tilde{R}$ whose kinematics are governed by $\mathcal{H}$, irrespective of the value of $q$.  Next, with a slight abuse of notation, when $\theta \coloneqq \vex{\log(R)}$, $R \in \mathcal{O}$ is equivalent to saying $\theta\in\mathcal{O}$, where $\mathcal{O}$ is any set. Define $\Tilde{\theta}(t) \coloneqq \vex{\log{(\Tilde{R}(t))}}$. 
  Since almost globally asymptotic stability is proved for $\Tilde{R}(t) \in C_0 \subset \SOtwo \backslash \mathbb{U}_0$  in \cite{mahony_complementaryFilter}, we have the following $\mathcal{KL}$ bound in $C_0$, where $\beta_0$ is a class $\mathcal{KL}$ function.
    \begin{align}\label{eq:KL_C0}
        |\Tilde{\theta}(t, j)|_\mathcal{A} \leq \beta_0\brackets{|\tilde{\theta}(0, 0)|_\mathcal{A}, t+j}, \quad \forall \tilde{\theta}(0, 0) \in C_0
    \end{align}
    We consider the four possible cases where $\tilde{R}(0, 0)$ can lie. When $\tilde{R}(0, 0)\in C_0$  we have asymptotic stability with the bounds in \eqref{eq:KL_C0}.
    When $\tilde{R}(0,0)\in D_1$, the jump at $(t,j) = (0,0)$ puts $\tilde{R}(0, 1) \in C_0$ immediately. Thus, we have the following bound similar to \eqref{eq:KL_C0}.
    \begin{align}\label{eq:KL_D1}
        |\Tilde{\theta}(t, j)|_\mathcal{A} \leq \beta_0\brackets{|\tilde{\theta}(0, 0)|_\mathcal{A}, t+j}, \quad \forall \tilde{\theta}(0, 0) \in D_1
    \end{align}
    Since $\beta_0\in \mathcal{KL}$, $\exists K_0$ which is a class $\mathcal{K}$ function such that 
    \begin{align}\label{eq:K_bound_D1}
        |\tilde\theta(t,j)|_\mathcal{A} \leq K_0(|\tilde\theta(0,0)|_\mathcal{A}), \quad \forall \tilde{\theta}(0,0) \in D_1
    \end{align}
    
    Next, when $\Tilde{R}(0,0)\in C_1$, using \eqref{eq:Rtilde_dyn}, we have a linear decay of $\tilde\theta(t, 0)$ as below. 
    \begin{align}
        \tilde{\theta}(t, 0) = \tilde{\theta}(0,0) - t
    \end{align}
    \textcolor{red}{For $\tilde{\theta}\in C_1$, we want $\tilde{\theta}$ to enter $D_1$ such that the inequality in \eqref{eq:KL_D1} holds for $\tilde\theta(0,0)\in C_1$.}
    \begin{align*}
    |\tilde{\theta}(t, j)|_{D_1} = \inf_{\theta\in D_1} |\tilde{\theta}(0,0) - t - {\theta}|
    \end{align*}
    Due to the construction of $C_1$ and $D_1$, $\tilde\theta(0,0) > \theta$ when $\theta\in D_1$. The infimum is first achieved at $\theta_1 \in [0,\pi]$ such that $\Psi\brackets{\exp(\tilde{{\theta}}_1)} = c_1$. Once this is achieved, $\tilde{\theta}(t,j)\in D_1$ and the jump to $C_0$ takes place. Thus, $\tilde{\theta}(0,0)\in C_1$ enters $D_1$ in finite time $T$, where $T \coloneqq \tilde\theta(0,0) - \theta_1$. For $t \leq T$, we have the following.
    \begin{align}
        |\tilde{\theta}(t, j)|_{D_1} &= |\tilde\theta(0,0) - \theta_1 - t|\nonumber\\
        & \leq |\tilde\theta(0,0)|_{D_1} + t
    \end{align}
    \textcolor{red}{Define a class $\mathcal{K}$ function $K_1(|\tilde\theta(0,0)|) \coloneqq |\tilde\theta(0,0)| + t$}. \textblue{This is not a class K function. Prove this again}. For the purpose of the function $K_1$, $t$ is a constant parameter. Thus, we have the following. 
    \begin{align}\label{eq:K_bound_C1}
        |\tilde\theta (t, j)|_{D_1} \leq K_1(|\tilde\theta(0,0)|_{D_1}), \quad \forall \tilde{\theta}(0,0) \in C_1
    \end{align}
    Now, for two class $\mathcal{K}$ functions $K_a$ and $K_b$, there must exist a class $\mathcal{K}$ function $K_c$ such that $K_a$ and $K_b$ are upper bounded by $K_c$. Using this and \eqref{eq:K_bound_C1}, \eqref{eq:K_bound_D1}, there must exist a class $\mathcal{K}$ function $K$ such that we get the following.
    \begin{align}\label{eq:K_bound_final_C1}
        |\tilde{\theta}(t, j)|_\mathcal{A}\leq K(|\tilde{\theta}(0,0)|_{\mathcal{A}}), \quad \forall \tilde\theta(0,0) \in C_1
    \end{align}
    Now, since $\tilde\theta(t,j)$ reaches $D_1$ in finite time $T$, we have the following bound with $\beta\in \mathcal{KL}$.
    \begin{align}\label{eq:KL_C1}
        |\Tilde{\theta}(t, j)|_\mathcal{A} \leq \beta\brackets{|\tilde{\theta}(0, 0)|_\mathcal{A}, t+j}, \quad \forall \tilde{\theta}(0, 0) \in C_1
    \end{align}

Finally, if $\tilde{R}(0,0)\in D_0$, $\tilde{R}(0,1)\in C_1$ and \eqref{eq:KL_C1} holds. Thus, we have a similar $\mathcal{KL}$ bound for $\tilde{R}(0,0)\in D_0$ as shown below. 
\begin{align}\label{eq:KL_D0}
    |\Tilde{\theta}(t, j)|_\mathcal{A} \leq \beta\brackets{|\tilde{\theta}(0, 0)|_\mathcal{A}, t+j}, \quad \forall \tilde{\theta}(0, 0) \in D_0
\end{align}
This translates to the following global $\mathcal{KL}$ bound on the state error. 
\begin{align}\label{eq:KL_global}
|\Tilde{\theta}(t, j)|_\mathcal{A} \leq \beta\brackets{|\tilde{\theta}(0, 0)|_\mathcal{A}, t+j}, \quad \forall \tilde{\theta}(0, 0) \in \SOtwo
\end{align}
Hence, the global asymptotic stability of $\mathcal{A}$ is proved. This proves $\eqref{(3)}$.
\end{proof}

\section{Numerical Simulations}
We consider $\Omega(t) = \sin\brackets{2\pi t/5}.$ For a simulation time step of $h=0.01 \sec$, $R(t) = R(0)\exp\brackets{h\Omega(t)_\times}$, with $R(0) = I \in \SOtwo$. The estimate $\hat{R}$ is initialized with $\hat{R}(0) = \log\brackets{\pi_\times}$. This results in $\tilde{R}\in \mathbb{U}_0$. For the hybrid observer, $c_0 = 3\pi/4$ and $c_1 = 2\pi/3$. The state $q$ is such that $q(0) = 0$. The following results are obtained.



\begin{figure}[!h]
\centering
\begin{minipage}{.49\textwidth}
  % \centering
  \flushleft
  \includegraphics[width=1.0\linewidth]{Figures/errFunction.jpg}
  \captionof{figure}{The error function $\Psi(\Tilde{R})$ goes from its maximum value of $2$ to $0$ asymptotically.}
  \label{fig:test1}
\end{minipage}
\begin{minipage}{.49\textwidth}
  % \centering
  \flushleft
  \includegraphics[width=1.0\linewidth]{Figures/errAngle.jpg}
  \captionof{figure}{The error angle $\Tilde{\theta}$ goes from its maximum value of $\pi$ to $0$ asymptotically. Linear decay is observed when $\tilde{\theta}\in C_1$. }
  \label{fig:test2}
\end{minipage}
\end{figure}

\begin{figure}[!h]
% \centering
\begin{minipage}{.49\textwidth}
  \includegraphics[width=1.0\linewidth]{Figures/est_track_true_angle.jpg}
  \captionof{figure}{The estimates track the true values of $\theta$.}
  \label{fig:tracking}
\end{minipage}
\begin{minipage}{.49\textwidth}
  % \centering
  \includegraphics[width=1.0\linewidth]{Figures/q_vs_time.jpg}
  \captionof{figure}{$q$ vs time. Two jumps occur when initial conditions start from $D_0$.}
  \label{fig:q_vs_time}
\end{minipage}
\end{figure}

% \begin{figure}
%      \centering
%      \begin{subfigure}[b]{0.3\textwidth}
%          \centering
%          \includegraphics[width=\textwidth]{Figures/errorFunction.png}
%          \caption{Error function $\Psi(\tilde{R})$ vs time.$\Psi(\tilde{R})$ goes from its maximum value of $2$ to $0$ asymptotically.}
%          \label{fig:err_function}
%      \end{subfigure}
%      \hfill
%      \begin{subfigure}[b]{0.3\textwidth}
%          \centering
%          \includegraphics[width=\textwidth]{Figures/errorAngle.png}
%          \caption{The error angle $\tilde{\theta}$ goes from $\pi$ to $0$ asymptotically}. 
%          \label{fig:err_angle}
%      \end{subfigure}
% \end{figure}

\section{Considering perturbations}

\subsection{Noise in gyroscope measurements}
In this section, we move away from Assumption 1 and consider a bounded measurement noise in the measurements of $\Omega$, i.e. $\Omega^y = \Omega + \eta_w$, with the bound on $\eta_w$ defined as $\etabound \coloneqq \sup_{t\geq 0}\|\eta_w\|$.

\begin{assumption} \label{ass:assumption2}
    The attitude measurements are assumed to be perfect, i.e. $R^y = R$. The angular velocity measurements have a bounded error $ \eta_w$ such that $\Omega^y = \Omega + \eta_w$, where $\|\eta_w\|\leq \etabound$ for all $t\geq 0$.
\end{assumption}

Under these noisy measurements, the perturbed hybrid observer, denoted as $\hat{\mathcal{H}}_w = ({C}^w_K,{F}^w_K, {D}^w_K,  {G}^w_K, \zeta^w )$, is now given as follows. Since only $\zeta_K$ is affected by the measurement noise, the flow maps and the flow sets for the perturbed hybrid observer remain the same. Thus,
\[{C}^w_K = C_K, \quad {F}^w_K = F_K, \quad  {D}^w_K = D_K, \quad  {G}^w_K = G_K.\]



\bibliographystyle{plain}
\bibliography{ref}
\end{document}
